\documentclass[a4paper]{exam}

\usepackage{amsmath,amssymb}
\usepackage{geometry}
\usepackage{graphicx}
\usepackage{hyperref}
\usepackage{titling}

% Header and footer.
\pagestyle{headandfoot}
\runningheadrule
\runningfootrule
\runningheader{CS 212, Fall 2022}{WC 10: Complexity Classes}{\theauthor}
\runningfooter{}{Page \thepage\ of \numpages}{}
\firstpageheader{}{}{}

\printanswers

\title{Weekly Challenge 10: Complexity Classes}
\author{ungraded} % <=== replace with your student ID, e.g. xy012345
\date{CS 212 Nature of Computation\\Habib University\\Fall 2022}

\qformat{{\large\bf \thequestion. \thequestiontitle}\hfill}
\boxedpoints

\begin{document}
\maketitle

\begin{questions}
  
\titledquestion{Checking Primality}

Explain succinctly why the language recognized by the following Turing Machine, M, does not belong to $\mathbf{P}$ (assume the input is represented in binary)

$M = $ On input $n$:
\begin{enumerate}
\item Check if 2 divides $n$, if so \textit{Reject}.
\item Otherwise, repeat step 1 for all numbers less than $n$. That is, check if 3 divides $n$. If so \textit{Reject}, otherwise check if 4 divides $n$, if so Reject, and so on.
\item If all numbers less than $n$ have been checked, \textit{Accept}.  
\end{enumerate}
\begin{solution}
    % Enter your solution here.
  \end{solution}
\end{questions}
\end{document}

%%% Local Variables:
%%% mode: latex
%%% TeX-master: t
%%% End:
